\chapter{Introduction}
\label{intro}

The ``Python library'' contains several different kinds of components.

It contains data types that would normally be considered part of the
``core'' of a language, such as numbers and lists.  For these types,
the Python language core defines the form of literals and places some
constraints on their semantics, but does not fully define the
semantics.  (On the other hand, the language core does define
syntactic properties like the spelling and priorities of operators.)

The library also contains built-in functions and exceptions ---
objects that can be used by all Python code without the need of an
\keyword{import} statement.  Some of these are defined by the core
language, but many are not essential for the core semantics and are
only described here.

The bulk of the library, however, consists of a collection of modules.
There are many ways to dissect this collection.  Some modules are
written in C and built in to the Python interpreter; others are
written in Python and imported in source form.  Some modules provide
interfaces that are highly specific to Python, like printing a stack
trace; some provide interfaces that are specific to particular
operating systems, such as access to specific hardware; others provide
interfaces that are
specific to a particular application domain, like the World-Wide Web.
Some modules are available in all versions and ports of Python; others
are only available when the underlying system supports or requires
them; yet others are available only when a particular configuration
option was chosen at the time when Python was compiled and installed.

This manual is organized ``from the inside out:'' it first describes
the built-in data types, then the built-in functions and exceptions,
and finally the modules, grouped in chapters of related modules.  The
ordering of the chapters as well as the ordering of the modules within
each chapter is roughly from most relevant to least important.

This means that if you start reading this manual from the start, and
skip to the next chapter when you get bored, you will get a reasonable
overview of the available modules and application areas that are
supported by the Python library.  Of course, you don't \emph{have} to
read it like a novel --- you can also browse the table of contents (in
front of the manual), or look for a specific function, module or term
in the index (in the back).  And finally, if you enjoy learning about
random subjects, you choose a random page number (see module
\refmodule{random}) and read a section or two.  Regardless of the
order in which you read the sections of this manual, it helps to start 
with chapter \ref{builtin}, ``Built-in Types, Exceptions and
Functions,'' as the remainder of the manual assumes familiarity with
this material.

Let the show begin!
