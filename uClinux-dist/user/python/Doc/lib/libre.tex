\section{\module{re} ---
         Regular expression operations}
\declaremodule{standard}{re}
\moduleauthor{Andrew M. Kuchling}{amk1@bigfoot.com}
\moduleauthor{Fredrik Lundh}{effbot@telia.com}
\sectionauthor{Andrew M. Kuchling}{amk1@bigfoot.com}


\modulesynopsis{Regular expression search and match operations with a
                Perl-style expression syntax.}


This module provides regular expression matching operations similar to
those found in Perl.  Regular expression pattern strings may not
contain null bytes, but can specify the null byte using the
\code{\e\var{number}} notation.  Both patterns and strings to be
searched can be Unicode strings as well as 8-bit strings.  The
\module{re} module is always available.

Regular expressions use the backslash character (\character{\e}) to
indicate special forms or to allow special characters to be used
without invoking their special meaning.  This collides with Python's
usage of the same character for the same purpose in string literals;
for example, to match a literal backslash, one might have to write
\code{'\e\e\e\e'} as the pattern string, because the regular expression
must be \samp{\e\e}, and each backslash must be expressed as
\samp{\e\e} inside a regular Python string literal. 

The solution is to use Python's raw string notation for regular
expression patterns; backslashes are not handled in any special way in
a string literal prefixed with \character{r}.  So \code{r"\e n"} is a
two-character string containing \character{\e} and \character{n},
while \code{"\e n"} is a one-character string containing a newline.
Usually patterns will be expressed in Python code using this raw
string notation.

\strong{Implementation note:}
The \module{re}\refstmodindex{pre} module has two distinct
implementations: \module{sre} is the default implementation and
includes Unicode support, but may run into stack limitations for some
patterns.  Though this will be fixed for a future release of Python,
the older implementation (without Unicode support) is still available
as the \module{pre}\refstmodindex{pre} module.


\subsection{Regular Expression Syntax \label{re-syntax}}

A regular expression (or RE) specifies a set of strings that matches
it; the functions in this module let you check if a particular string
matches a given regular expression (or if a given regular expression
matches a particular string, which comes down to the same thing).

Regular expressions can be concatenated to form new regular
expressions; if \emph{A} and \emph{B} are both regular expressions,
then \emph{AB} is also an regular expression.  If a string \emph{p}
matches A and another string \emph{q} matches B, the string \emph{pq}
will match AB.  Thus, complex expressions can easily be constructed
from simpler primitive expressions like the ones described here.  For
details of the theory and implementation of regular expressions,
consult the Friedl book referenced below, or almost any textbook about
compiler construction.

A brief explanation of the format of regular expressions follows.  For
further information and a gentler presentation, consult the Regular
Expression HOWTO, accessible from \url{http://www.python.org/doc/howto/}.

Regular expressions can contain both special and ordinary characters.
Most ordinary characters, like \character{A}, \character{a}, or \character{0},
are the simplest regular expressions; they simply match themselves.  
You can concatenate ordinary characters, so \regexp{last} matches the
string \code{'last'}.  (In the rest of this section, we'll write RE's in
\regexp{this special style}, usually without quotes, and strings to be
matched \code{'in single quotes'}.)

Some characters, like \character{|} or \character{(}, are special.  Special
characters either stand for classes of ordinary characters, or affect
how the regular expressions around them are interpreted.

The special characters are:

\begin{list}{}{\leftmargin 0.7in \labelwidth 0.65in}

\item[\character{.}] (Dot.)  In the default mode, this matches any
character except a newline.  If the \constant{DOTALL} flag has been
specified, this matches any character including a newline.

\item[\character{\^}] (Caret.)  Matches the start of the string, and in
\constant{MULTILINE} mode also matches immediately after each newline.

\item[\character{\$}] Matches the end of the string, and in
\constant{MULTILINE} mode also matches before a newline.
\regexp{foo} matches both 'foo' and 'foobar', while the regular
expression \regexp{foo\$} matches only 'foo'.

\item[\character{*}] Causes the resulting RE to
match 0 or more repetitions of the preceding RE, as many repetitions
as are possible.  \regexp{ab*} will
match 'a', 'ab', or 'a' followed by any number of 'b's.

\item[\character{+}] Causes the
resulting RE to match 1 or more repetitions of the preceding RE.
\regexp{ab+} will match 'a' followed by any non-zero number of 'b's; it
will not match just 'a'.

\item[\character{?}] Causes the resulting RE to
match 0 or 1 repetitions of the preceding RE.  \regexp{ab?} will
match either 'a' or 'ab'.
\item[\code{*?}, \code{+?}, \code{??}] The \character{*}, \character{+}, and
\character{?} qualifiers are all \dfn{greedy}; they match as much text as
possible.  Sometimes this behaviour isn't desired; if the RE
\regexp{<.*>} is matched against \code{'<H1>title</H1>'}, it will match the
entire string, and not just \code{'<H1>'}.
Adding \character{?} after the qualifier makes it perform the match in
\dfn{non-greedy} or \dfn{minimal} fashion; as \emph{few} characters as
possible will be matched.  Using \regexp{.*?} in the previous
expression will match only \code{'<H1>'}.

\item[\code{\{\var{m},\var{n}\}}] Causes the resulting RE to match from
\var{m} to \var{n} repetitions of the preceding RE, attempting to
match as many repetitions as possible.  For example, \regexp{a\{3,5\}}
will match from 3 to 5 \character{a} characters.  Omitting \var{n}
specifies an infinite upper bound; you can't omit \var{m}.

\item[\code{\{\var{m},\var{n}\}?}] Causes the resulting RE to
match from \var{m} to \var{n} repetitions of the preceding RE,
attempting to match as \emph{few} repetitions as possible.  This is
the non-greedy version of the previous qualifier.  For example, on the
6-character string \code{'aaaaaa'}, \regexp{a\{3,5\}} will match 5
\character{a} characters, while \regexp{a\{3,5\}?} will only match 3
characters.

\item[\character{\e}] Either escapes special characters (permitting
you to match characters like \character{*}, \character{?}, and so
forth), or signals a special sequence; special sequences are discussed
below.

If you're not using a raw string to
express the pattern, remember that Python also uses the
backslash as an escape sequence in string literals; if the escape
sequence isn't recognized by Python's parser, the backslash and
subsequent character are included in the resulting string.  However,
if Python would recognize the resulting sequence, the backslash should
be repeated twice.  This is complicated and hard to understand, so
it's highly recommended that you use raw strings for all but the
simplest expressions.

\item[\code{[]}] Used to indicate a set of characters.  Characters can
be listed individually, or a range of characters can be indicated by
giving two characters and separating them by a \character{-}.  Special
characters are not active inside sets.  For example, \regexp{[akm\$]}
will match any of the characters \character{a}, \character{k},
\character{m}, or \character{\$}; \regexp{[a-z]}
will match any lowercase letter, and \code{[a-zA-Z0-9]} matches any
letter or digit.  Character classes such as \code{\e w} or \code{\e S}
(defined below) are also acceptable inside a range.  If you want to
include a \character{]} or a \character{-} inside a set, precede it with a
backslash, or place it as the first character.  The 
pattern \regexp{[]]} will match \code{']'}, for example.  

You can match the characters not within a range by \dfn{complementing}
the set.  This is indicated by including a
\character{\^} as the first character of the set; \character{\^} elsewhere will
simply match the \character{\^} character.  For example, \regexp{[{\^}5]}
will match any character except \character{5}.

\item[\character{|}]\code{A|B}, where A and B can be arbitrary REs,
creates a regular expression that will match either A or B.  An
arbitrary number of REs can be separated by the \character{|} in this
way.  This can be used inside groups (see below) as well.  REs
separated by \character{|} are tried from left to right, and the first
one that allows the complete pattern to match is considered the
accepted branch.  This means that if \code{A} matches, \code{B} will
never be tested, even if it would produce a longer overall match.  In
other words, the \character{|} operator is never greedy.  To match a
literal \character{|}, use \regexp{\e|}, or enclose it inside a
character class, as in \regexp{[|]}.

\item[\code{(...)}] Matches whatever regular expression is inside the
parentheses, and indicates the start and end of a group; the contents
of a group can be retrieved after a match has been performed, and can
be matched later in the string with the \regexp{\e \var{number}} special
sequence, described below.  To match the literals \character{(} or
\character{)}, use \regexp{\e(} or \regexp{\e)}, or enclose them
inside a character class: \regexp{[(] [)]}.

\item[\code{(?...)}] This is an extension notation (a \character{?}
following a \character{(} is not meaningful otherwise).  The first
character after the \character{?} 
determines what the meaning and further syntax of the construct is.
Extensions usually do not create a new group;
\regexp{(?P<\var{name}>...)} is the only exception to this rule.
Following are the currently supported extensions.

\item[\code{(?iLmsux)}] (One or more letters from the set \character{i},
\character{L}, \character{m}, \character{s}, \character{u},
\character{x}.)  The group matches the empty string; the letters set
the corresponding flags (\constant{re.I}, \constant{re.L},
\constant{re.M}, \constant{re.S}, \constant{re.U}, \constant{re.X})
for the entire regular expression.  This is useful if you wish to
include the flags as part of the regular expression, instead of
passing a \var{flag} argument to the \function{compile()} function.

Note that the \regexp{(?x)} flag changes how the expression is parsed.
It should be used first in the expression string, or after one or more
whitespace characters.  If there are non-whitespace characters before
the flag, the results are undefined.

\item[\code{(?:...)}] A non-grouping version of regular parentheses.
Matches whatever regular expression is inside the parentheses, but the
substring matched by the 
group \emph{cannot} be retrieved after performing a match or
referenced later in the pattern. 

\item[\code{(?P<\var{name}>...)}] Similar to regular parentheses, but
the substring matched by the group is accessible via the symbolic group
name \var{name}.  Group names must be valid Python identifiers.  A
symbolic group is also a numbered group, just as if the group were not
named.  So the group named 'id' in the example above can also be
referenced as the numbered group 1.

For example, if the pattern is
\regexp{(?P<id>[a-zA-Z_]\e w*)}, the group can be referenced by its
name in arguments to methods of match objects, such as \code{m.group('id')}
or \code{m.end('id')}, and also by name in pattern text
(e.g. \regexp{(?P=id)}) and replacement text (e.g. \code{\e g<id>}).

\item[\code{(?P=\var{name})}] Matches whatever text was matched by the
earlier group named \var{name}.

\item[\code{(?\#...)}] A comment; the contents of the parentheses are
simply ignored.

\item[\code{(?=...)}] Matches if \regexp{...} matches next, but doesn't
consume any of the string.  This is called a lookahead assertion.  For
example, \regexp{Isaac (?=Asimov)} will match \code{'Isaac~'} only if it's
followed by \code{'Asimov'}.

\item[\code{(?!...)}] Matches if \regexp{...} doesn't match next.  This
is a negative lookahead assertion.  For example,
\regexp{Isaac (?!Asimov)} will match \code{'Isaac~'} only if it's \emph{not}
followed by \code{'Asimov'}.

\item[\code{(?<=...)}] Matches if the current position in the string
is preceded by a match for \regexp{...} that ends at the current
position.  This is called a positive lookbehind assertion.
\regexp{(?<=abc)def} will match \samp{abcdef}, since the lookbehind
will back up 3 characters and check if the contained pattern matches.
The contained pattern must only match strings of some fixed length,
meaning that \regexp{abc} or \regexp{a|b} are allowed, but \regexp{a*}
isn't.

\item[\code{(?<!...)}] Matches if the current position in the string
is not preceded by a match for \regexp{...}.  This
is called a negative lookbehind assertion.  Similar to positive lookbehind
assertions, the contained pattern must only match strings of some
fixed length.

\end{list}

The special sequences consist of \character{\e} and a character from the
list below.  If the ordinary character is not on the list, then the
resulting RE will match the second character.  For example,
\regexp{\e\$} matches the character \character{\$}.

\begin{list}{}{\leftmargin 0.7in \labelwidth 0.65in}

\item[\code{\e \var{number}}] Matches the contents of the group of the
same number.  Groups are numbered starting from 1.  For example,
\regexp{(.+) \e 1} matches \code{'the the'} or \code{'55 55'}, but not
\code{'the end'} (note 
the space after the group).  This special sequence can only be used to
match one of the first 99 groups.  If the first digit of \var{number}
is 0, or \var{number} is 3 octal digits long, it will not be interpreted
as a group match, but as the character with octal value \var{number}.
Inside the \character{[} and \character{]} of a character class, all numeric
escapes are treated as characters. 

\item[\code{\e A}] Matches only at the start of the string.

\item[\code{\e b}] Matches the empty string, but only at the
beginning or end of a word.  A word is defined as a sequence of
alphanumeric characters, so the end of a word is indicated by
whitespace or a non-alphanumeric character.  Inside a character range,
\regexp{\e b} represents the backspace character, for compatibility with
Python's string literals.

\item[\code{\e B}] Matches the empty string, but only when it is
\emph{not} at the beginning or end of a word.

\item[\code{\e d}]Matches any decimal digit; this is
equivalent to the set \regexp{[0-9]}.

\item[\code{\e D}]Matches any non-digit character; this is
equivalent to the set \regexp{[{\^}0-9]}.

\item[\code{\e s}]Matches any whitespace character; this is
equivalent to the set \regexp{[ \e t\e n\e r\e f\e v]}.

\item[\code{\e S}]Matches any non-whitespace character; this is
equivalent to the set \regexp{[\^\ \e t\e n\e r\e f\e v]}.

\item[\code{\e w}]When the \constant{LOCALE} and \constant{UNICODE}
flags are not specified,
matches any alphanumeric character; this is equivalent to the set
\regexp{[a-zA-Z0-9_]}.  With \constant{LOCALE}, it will match the set
\regexp{[0-9_]} plus whatever characters are defined as letters for
the current locale.  If \constant{UNICODE} is set, this will match the
characters \regexp{[0-9_]} plus whatever is classified as alphanumeric
in the Unicode character properties database.

\item[\code{\e W}]When the \constant{LOCALE} and \constant{UNICODE}
flags are not specified, matches any non-alphanumeric character; this
is equivalent to the set \regexp{[{\^}a-zA-Z0-9_]}.   With
\constant{LOCALE}, it will match any character not in the set
\regexp{[0-9_]}, and not defined as a letter for the current locale.
If \constant{UNICODE} is set, this will match anything other than
\regexp{[0-9_]} and characters marked at alphanumeric in the Unicode
character properties database.

\item[\code{\e Z}]Matches only at the end of the string.

\item[\code{\e \e}] Matches a literal backslash.

\end{list}


\subsection{Matching vs. Searching \label{matching-searching}}
\sectionauthor{Fred L. Drake, Jr.}{fdrake@acm.org}

Python offers two different primitive operations based on regular
expressions: match and search.  If you are accustomed to Perl's
semantics, the search operation is what you're looking for.  See the
\function{search()} function and corresponding method of compiled
regular expression objects.

Note that match may differ from search using a regular expression
beginning with \character{\^}: \character{\^} matches only at the
start of the string, or in \constant{MULTILINE} mode also immediately
following a newline.  The ``match'' operation succeeds only if the
pattern matches at the start of the string regardless of mode, or at
the starting position given by the optional \var{pos} argument
regardless of whether a newline precedes it.

% Examples from Tim Peters:
\begin{verbatim}
re.compile("a").match("ba", 1)           # succeeds
re.compile("^a").search("ba", 1)         # fails; 'a' not at start
re.compile("^a").search("\na", 1)        # fails; 'a' not at start
re.compile("^a", re.M).search("\na", 1)  # succeeds
re.compile("^a", re.M).search("ba", 1)   # fails; no preceding \n
\end{verbatim}


\subsection{Module Contents}
\nodename{Contents of Module re}

The module defines the following functions and constants, and an exception:


\begin{funcdesc}{compile}{pattern\optional{, flags}}
  Compile a regular expression pattern into a regular expression
  object, which can be used for matching using its \function{match()} and
  \function{search()} methods, described below.  

  The expression's behaviour can be modified by specifying a
  \var{flags} value.  Values can be any of the following variables,
  combined using bitwise OR (the \code{|} operator).

The sequence

\begin{verbatim}
prog = re.compile(pat)
result = prog.match(str)
\end{verbatim}

is equivalent to

\begin{verbatim}
result = re.match(pat, str)
\end{verbatim}

but the version using \function{compile()} is more efficient when the
expression will be used several times in a single program.
%(The compiled version of the last pattern passed to
%\function{regex.match()} or \function{regex.search()} is cached, so
%programs that use only a single regular expression at a time needn't
%worry about compiling regular expressions.)
\end{funcdesc}

\begin{datadesc}{I}
\dataline{IGNORECASE}
Perform case-insensitive matching; expressions like \regexp{[A-Z]} will match
lowercase letters, too.  This is not affected by the current locale.
\end{datadesc}

\begin{datadesc}{L}
\dataline{LOCALE}
Make \regexp{\e w}, \regexp{\e W}, \regexp{\e b}, and
\regexp{\e B} dependent on the current locale. 
\end{datadesc}

\begin{datadesc}{M}
\dataline{MULTILINE}
When specified, the pattern character \character{\^} matches at the
beginning of the string and at the beginning of each line
(immediately following each newline); and the pattern character
\character{\$} matches at the end of the string and at the end of each line
(immediately preceding each newline).
By default, \character{\^} matches only at the beginning of the string, and
\character{\$} only at the end of the string and immediately before the
newline (if any) at the end of the string. 
\end{datadesc}

\begin{datadesc}{S}
\dataline{DOTALL}
Make the \character{.} special character match any character at all,
including a newline; without this flag, \character{.} will match
anything \emph{except} a newline.
\end{datadesc}

\begin{datadesc}{U}
\dataline{UNICODE}
Make \regexp{\e w}, \regexp{\e W}, \regexp{\e b}, and
\regexp{\e B} dependent on the Unicode character properties database.
\versionadded{2.0}
\end{datadesc}

\begin{datadesc}{X}
\dataline{VERBOSE}
This flag allows you to write regular expressions that look nicer.
Whitespace within the pattern is ignored, 
except when in a character class or preceded by an unescaped
backslash, and, when a line contains a \character{\#} neither in a character
class or preceded by an unescaped backslash, all characters from the
leftmost such \character{\#} through the end of the line are ignored.
% XXX should add an example here
\end{datadesc}


\begin{funcdesc}{search}{pattern, string\optional{, flags}}
  Scan through \var{string} looking for a location where the regular
  expression \var{pattern} produces a match, and return a
  corresponding \class{MatchObject} instance.
  Return \code{None} if no
  position in the string matches the pattern; note that this is
  different from finding a zero-length match at some point in the string.
\end{funcdesc}

\begin{funcdesc}{match}{pattern, string\optional{, flags}}
  If zero or more characters at the beginning of \var{string} match
  the regular expression \var{pattern}, return a corresponding
  \class{MatchObject} instance.  Return \code{None} if the string does not
  match the pattern; note that this is different from a zero-length
  match.

  \strong{Note:}  If you want to locate a match anywhere in
  \var{string}, use \method{search()} instead.
\end{funcdesc}

\begin{funcdesc}{split}{pattern, string\optional{, maxsplit\code{ = 0}}}
  Split \var{string} by the occurrences of \var{pattern}.  If
  capturing parentheses are used in \var{pattern}, then the text of all
  groups in the pattern are also returned as part of the resulting list.
  If \var{maxsplit} is nonzero, at most \var{maxsplit} splits
  occur, and the remainder of the string is returned as the final
  element of the list.  (Incompatibility note: in the original Python
  1.5 release, \var{maxsplit} was ignored.  This has been fixed in
  later releases.)

\begin{verbatim}
>>> re.split('\W+', 'Words, words, words.')
['Words', 'words', 'words', '']
>>> re.split('(\W+)', 'Words, words, words.')
['Words', ', ', 'words', ', ', 'words', '.', '']
>>> re.split('\W+', 'Words, words, words.', 1)
['Words', 'words, words.']
\end{verbatim}

  This function combines and extends the functionality of
  the old \function{regsub.split()} and \function{regsub.splitx()}.
\end{funcdesc}

\begin{funcdesc}{findall}{pattern, string}
Return a list of all non-overlapping matches of \var{pattern} in
\var{string}.  If one or more groups are present in the pattern,
return a list of groups; this will be a list of tuples if the pattern
has more than one group.  Empty matches are included in the result.
\versionadded{1.5.2}
\end{funcdesc}

\begin{funcdesc}{sub}{pattern, repl, string\optional{, count\code{ = 0}}}
Return the string obtained by replacing the leftmost non-overlapping
occurrences of \var{pattern} in \var{string} by the replacement
\var{repl}.  If the pattern isn't found, \var{string} is returned
unchanged.  \var{repl} can be a string or a function; if a function,
it is called for every non-overlapping occurrence of \var{pattern}.
The function takes a single match object argument, and returns the
replacement string.  For example:

\begin{verbatim}
>>> def dashrepl(matchobj):
....    if matchobj.group(0) == '-': return ' '
....    else: return '-'
>>> re.sub('-{1,2}', dashrepl, 'pro----gram-files')
'pro--gram files'
\end{verbatim}

The pattern may be a string or a 
regex object; if you need to specify
regular expression flags, you must use a regex object, or use
embedded modifiers in a pattern; e.g.
\samp{sub("(?i)b+", "x", "bbbb BBBB")} returns \code{'x x'}.

The optional argument \var{count} is the maximum number of pattern
occurrences to be replaced; \var{count} must be a non-negative integer, and
the default value of 0 means to replace all occurrences.

Empty matches for the pattern are replaced only when not adjacent to a
previous match, so \samp{sub('x*', '-', 'abc')} returns \code{'-a-b-c-'}.

If \var{repl} is a string, any backslash escapes in it are processed.
That is, \samp{\e n} is converted to a single newline character,
\samp{\e r} is converted to a linefeed, and so forth.  Unknown escapes
such as \samp{\e j} are left alone.  Backreferences, such as \samp{\e 6}, are
replaced with the substring matched by group 6 in the pattern. 

In addition to character escapes and backreferences as described
above, \samp{\e g<name>} will use the substring matched by the group
named \samp{name}, as defined by the \regexp{(?P<name>...)} syntax.
\samp{\e g<number>} uses the corresponding group number; \samp{\e
g<2>} is therefore equivalent to \samp{\e 2}, but isn't ambiguous in a
replacement such as \samp{\e g<2>0}.  \samp{\e 20} would be
interpreted as a reference to group 20, not a reference to group 2
followed by the literal character \character{0}.  
\end{funcdesc}

\begin{funcdesc}{subn}{pattern, repl, string\optional{, count\code{ = 0}}}
Perform the same operation as \function{sub()}, but return a tuple
\code{(\var{new_string}, \var{number_of_subs_made})}.
\end{funcdesc}

\begin{funcdesc}{escape}{string}
  Return \var{string} with all non-alphanumerics backslashed; this is
  useful if you want to match an arbitrary literal string that may have
  regular expression metacharacters in it.
\end{funcdesc}

\begin{excdesc}{error}
  Exception raised when a string passed to one of the functions here
  is not a valid regular expression (e.g., unmatched parentheses) or
  when some other error occurs during compilation or matching.  It is
  never an error if a string contains no match for a pattern.
\end{excdesc}


\subsection{Regular Expression Objects \label{re-objects}}

Compiled regular expression objects support the following methods and
attributes:

\begin{methoddesc}[RegexObject]{search}{string\optional{, pos\optional{,
                                        endpos}}}
  Scan through \var{string} looking for a location where this regular
  expression produces a match, and return a
  corresponding \class{MatchObject} instance.  Return \code{None} if no
  position in the string matches the pattern; note that this is
  different from finding a zero-length match at some point in the string.
  
  The optional \var{pos} and \var{endpos} parameters have the same
  meaning as for the \method{match()} method.
\end{methoddesc}

\begin{methoddesc}[RegexObject]{match}{string\optional{, pos\optional{,
                                       endpos}}}
  If zero or more characters at the beginning of \var{string} match
  this regular expression, return a corresponding
  \class{MatchObject} instance.  Return \code{None} if the string does not
  match the pattern; note that this is different from a zero-length
  match.

  \strong{Note:}  If you want to locate a match anywhere in
  \var{string}, use \method{search()} instead.

  The optional second parameter \var{pos} gives an index in the string
  where the search is to start; it defaults to \code{0}.  This is not
  completely equivalent to slicing the string; the \code{'\^'} pattern
  character matches at the real beginning of the string and at positions
  just after a newline, but not necessarily at the index where the search
  is to start.

  The optional parameter \var{endpos} limits how far the string will
  be searched; it will be as if the string is \var{endpos} characters
  long, so only the characters from \var{pos} to \var{endpos} will be
  searched for a match.
\end{methoddesc}

\begin{methoddesc}[RegexObject]{split}{string\optional{,
                                       maxsplit\code{ = 0}}}
Identical to the \function{split()} function, using the compiled pattern.
\end{methoddesc}

\begin{methoddesc}[RegexObject]{findall}{string}
Identical to the \function{findall()} function, using the compiled pattern.
\end{methoddesc}

\begin{methoddesc}[RegexObject]{sub}{repl, string\optional{, count\code{ = 0}}}
Identical to the \function{sub()} function, using the compiled pattern.
\end{methoddesc}

\begin{methoddesc}[RegexObject]{subn}{repl, string\optional{,
                                      count\code{ = 0}}}
Identical to the \function{subn()} function, using the compiled pattern.
\end{methoddesc}


\begin{memberdesc}[RegexObject]{flags}
The flags argument used when the regex object was compiled, or
\code{0} if no flags were provided.
\end{memberdesc}

\begin{memberdesc}[RegexObject]{groupindex}
A dictionary mapping any symbolic group names defined by 
\regexp{(?P<\var{id}>)} to group numbers.  The dictionary is empty if no
symbolic groups were used in the pattern.
\end{memberdesc}

\begin{memberdesc}[RegexObject]{pattern}
The pattern string from which the regex object was compiled.
\end{memberdesc}


\subsection{Match Objects \label{match-objects}}

\class{MatchObject} instances support the following methods and attributes:

\begin{methoddesc}[MatchObject]{expand}{template}
 Return the string obtained by doing backslash substitution on the
template string \var{template}, as done by the \method{sub()} method.
Escapes such as \samp{\e n} are converted to the appropriate
characters, and numeric backreferences (\samp{\e 1}, \samp{\e 2}) and named
backreferences (\samp{\e g<1>}, \samp{\e g<name>}) are replaced by the contents of the
corresponding group.
\end{methoddesc}

\begin{methoddesc}[MatchObject]{group}{\optional{group1, \moreargs}}
Returns one or more subgroups of the match.  If there is a single
argument, the result is a single string; if there are
multiple arguments, the result is a tuple with one item per argument.
Without arguments, \var{group1} defaults to zero (i.e. the whole match
is returned).
If a \var{groupN} argument is zero, the corresponding return value is the
entire matching string; if it is in the inclusive range [1..99], it is
the string matching the the corresponding parenthesized group.  If a
group number is negative or larger than the number of groups defined
in the pattern, an \exception{IndexError} exception is raised.
If a group is contained in a part of the pattern that did not match,
the corresponding result is \code{-1}.  If a group is contained in a 
part of the pattern that matched multiple times, the last match is
returned.

If the regular expression uses the \regexp{(?P<\var{name}>...)} syntax,
the \var{groupN} arguments may also be strings identifying groups by
their group name.  If a string argument is not used as a group name in 
the pattern, an \exception{IndexError} exception is raised.

A moderately complicated example:

\begin{verbatim}
m = re.match(r"(?P<int>\d+)\.(\d*)", '3.14')
\end{verbatim}

After performing this match, \code{m.group(1)} is \code{'3'}, as is
\code{m.group('int')}, and \code{m.group(2)} is \code{'14'}.
\end{methoddesc}

\begin{methoddesc}[MatchObject]{groups}{\optional{default}}
Return a tuple containing all the subgroups of the match, from 1 up to
however many groups are in the pattern.  The \var{default} argument is
used for groups that did not participate in the match; it defaults to
\code{None}.  (Incompatibility note: in the original Python 1.5
release, if the tuple was one element long, a string would be returned
instead.  In later versions (from 1.5.1 on), a singleton tuple is
returned in such cases.)
\end{methoddesc}

\begin{methoddesc}[MatchObject]{groupdict}{\optional{default}}
Return a dictionary containing all the \emph{named} subgroups of the
match, keyed by the subgroup name.  The \var{default} argument is
used for groups that did not participate in the match; it defaults to
\code{None}.
\end{methoddesc}

\begin{methoddesc}[MatchObject]{start}{\optional{group}}
\funcline{end}{\optional{group}}
Return the indices of the start and end of the substring
matched by \var{group}; \var{group} defaults to zero (meaning the whole
matched substring).
Return \code{-1} if \var{group} exists but
did not contribute to the match.  For a match object
\var{m}, and a group \var{g} that did contribute to the match, the
substring matched by group \var{g} (equivalent to
\code{\var{m}.group(\var{g})}) is

\begin{verbatim}
m.string[m.start(g):m.end(g)]
\end{verbatim}

Note that
\code{m.start(\var{group})} will equal \code{m.end(\var{group})} if
\var{group} matched a null string.  For example, after \code{\var{m} =
re.search('b(c?)', 'cba')}, \code{\var{m}.start(0)} is 1,
\code{\var{m}.end(0)} is 2, \code{\var{m}.start(1)} and
\code{\var{m}.end(1)} are both 2, and \code{\var{m}.start(2)} raises
an \exception{IndexError} exception.
\end{methoddesc}

\begin{methoddesc}[MatchObject]{span}{\optional{group}}
For \class{MatchObject} \var{m}, return the 2-tuple
\code{(\var{m}.start(\var{group}), \var{m}.end(\var{group}))}.
Note that if \var{group} did not contribute to the match, this is
\code{(-1, -1)}.  Again, \var{group} defaults to zero.
\end{methoddesc}

\begin{memberdesc}[MatchObject]{pos}
The value of \var{pos} which was passed to the
\function{search()} or \function{match()} function.  This is the index into
the string at which the regex engine started looking for a match. 
\end{memberdesc}

\begin{memberdesc}[MatchObject]{endpos}
The value of \var{endpos} which was passed to the
\function{search()} or \function{match()} function.  This is the index into
the string beyond which the regex engine will not go.
\end{memberdesc}

\begin{memberdesc}[MatchObject]{re}
The regular expression object whose \method{match()} or
\method{search()} method produced this \class{MatchObject} instance.
\end{memberdesc}

\begin{memberdesc}[MatchObject]{string}
The string passed to \function{match()} or \function{search()}.
\end{memberdesc}

\begin{seealso}
\seetext{Jeffrey Friedl, \citetitle{Mastering Regular Expressions},
O'Reilly.  The Python material in this book dates from before the
\module{re} module, but it covers writing good regular expression
patterns in great detail.}
\end{seealso}

