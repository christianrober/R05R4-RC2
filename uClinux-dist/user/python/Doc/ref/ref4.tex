\chapter{Execution model \label{execmodel}}
\index{execution model}

\section{Code blocks, execution frames, and namespaces \label{execframes}}
\index{code block}
\index{namespace}
\indexii{execution}{frame}

A \dfn{code block}\indexii{code}{block} is a piece
of Python program text that can be executed as a unit, such as a
module, a class definition or a function body.  Some code blocks (like
modules) are normally executed only once, others (like function
bodies) may be executed many times.  Code blocks may textually contain
other code blocks.  Code blocks may invoke other code blocks (that may
or may not be textually contained in them) as part of their execution,
e.g., by invoking (calling) a function.

The following are code blocks: A module is a code block.  A function
body is a code block.  A class definition is a code block.  Each
command typed interactively is a separate code block; a script file (a
file given as standard input to the interpreter or specified on the
interpreter command line the first argument) is a code block; a script
command (a command specified on the interpreter command line with the
`\strong{-c}' option) is a code block.  The file read by the built-in
function \function{execfile()} is a code block.  The string argument
passed to the built-in function \function{eval()} and to the
\keyword{exec} statement is a code block.  And finally, the expression
read and evaluated by the built-in function \function{input()} is a
code block.

A code block is executed in an execution frame.  An \dfn{execution
frame}\indexii{execution}{frame} contains some administrative
information (used for debugging), determines where and how execution
continues after the code block's execution has completed, and (perhaps
most importantly) defines two namespaces, the local and the global
namespace, that affect execution of the code block.

A \dfn{namespace}\index{namespace} is a mapping from names
(identifiers) to objects.  A particular namespace may be referenced by
more than one execution frame, and from other places as well.  Adding
a name to a namespace is called \dfn{binding}\indexii{binding}{name} a
name (to an object); changing the mapping of a name is called
\dfn{rebinding}\indexii{rebinding}{name}; removing a name is
\dfn{unbinding}\indexii{unbinding}{name}.  Namespaces are functionally
equivalent to dictionaries (and often implemented as dictionaries).

The \dfn{local namespace}\indexii{local}{namespace} of an execution
frame determines the default place where names are defined and
searched.  The
\dfn{global namespace}\indexii{global}{namespace} determines the place
where names listed in \keyword{global}\stindex{global} statements are
defined and searched, and where names that are not bound anywhere in
the current code block are searched.

Whether a name is local or global in a code block is determined by
static inspection of the source text for the code block: in the
absence of \keyword{global} statements, a name that is bound anywhere
in the code block is local in the entire code block; all other names
are considered global.  The \keyword{global} statement forces global
interpretation of selected names throughout the code block.  The
following constructs bind names: formal parameters to functions,
\keyword{import} statements, class and function definitions (these
bind the class or function name in the defining block), and targets
that are identifiers if occurring in an assignment, \keyword{for} loop
header, or in the second position of an \keyword{except} clause
header.  Local names are searched only on the local namespace; global
names are searched only in the global and built-in
namespace.\footnote{
  If the code block contains \keyword{exec} statements or the
  construct ``\samp{from \ldots import *}'', the semantics of local
  names change: local name lookup first searches the local namespace,
  then the global namespace and the built-in namespace.}

A target occurring in a \keyword{del} statement is also considered bound
for this purpose (though the actual semantics are to ``unbind'' the
name).

When a global name is not found in the global namespace, it is
searched in the built-in namespace (which is actually the global
namespace of the module
\module{__builtin__}\refbimodindex{__builtin__}).  The built-in
namespace associated with the execution of a code block is actually
found by looking up the name \code{__builtins__} is its global
namespace; this should be a dictionary or a module (in the latter case
its dictionary is used).  Normally, the \code{__builtins__} namespace
is the dictionary of the built-in module \module{__builtin__} (note:
no `s'); if it isn't, restricted
execution\indexii{restricted}{execution} mode is in effect.  When a 
name is not found at all, a
\exception{NameError}\withsubitem{(built-in
exception)}{\ttindex{NameError}} exception is raised.
\stindex{from}
\stindex{exec}
\stindex{global}

The following table lists the meaning of the local and global
namespace for various types of code blocks.  The namespace for a
particular module is automatically created when the module is first
imported (i.e., when it is loaded).  Note that in almost all cases,
the global namespace is the namespace of the containing module ---
scopes in Python do not nest!

\begin{tableiv}{l|l|l|l}{textrm}
  {Code block type}{Global namespace}{Local namespace}{Notes}
  \lineiv{Module}
         {n.s. for this module}
         {same as global}{}
  \lineiv{Script (file or command)}
         {n.s. for \module{__main__}\refbimodindex{__main__}}
         {same as global}{(1)}
  \lineiv{Interactive command}
         {n.s. for \module{__main__}\refbimodindex{__main__}}
         {same as global}{}
  \lineiv{Class definition}
         {global n.s. of containing block}
         {new n.s.}{}
  \lineiv{Function body}
         {global n.s. of containing block}
         {new n.s.}{(2)}
  \lineiv{String passed to \keyword{exec} statement}
         {global n.s. of containing block}
         {local n.s. of containing block}{(2), (3)}
  \lineiv{String passed to \function{eval()}}
         {global n.s. of caller}
         {local n.s. of caller}{(2), (3)}
  \lineiv{File read by \function{execfile()}}
         {global n.s. of caller}
         {local n.s. of caller}{(2), (3)}
  \lineiv{Expression read by \function{input()}}
         {global n.s. of caller}
         {local n.s. of caller}{}
\end{tableiv}

Notes:

\begin{description}

\item[n.s.] means \emph{namespace}

\item[(1)] The main module for a script is always called
\module{__main__}; ``the filename don't enter into it.''

\item[(2)] The global and local namespace for these can be
overridden with optional extra arguments.

\item[(3)] The \keyword{exec} statement and the \function{eval()} and
\function{execfile()} functions have optional arguments to override
the global and local namespace.  If only one namespace is specified,
it is used for both.

\end{description}

The built-in functions \function{globals()} and \function{locals()} returns a
dictionary representing the current global and local namespace,
respectively.  The effect of modifications to this dictionary on the
namespace are undefined.\footnote{
  The current implementations return the dictionary actually used to
  implement the namespace, \emph{except} for functions, where the
  optimizer may cause the local namespace to be implemented
  differently, and \function{locals()} returns a read-only
  dictionary.}


\section{Exceptions \label{exceptions}}
\index{exception}

Exceptions are a means of breaking out of the normal flow of control
of a code block in order to handle errors or other exceptional
conditions.  An exception is
\emph{raised}\index{raise an exception} at the point where the error
is detected; it may be \emph{handled}\index{handle an exception} by
the surrounding code block or by any code block that directly or
indirectly invoked the code block where the error occurred.
\index{exception handler}
\index{errors}
\index{error handling}

The Python interpreter raises an exception when it detects a run-time
error (such as division by zero).  A Python program can also
explicitly raise an exception with the \keyword{raise} statement.
Exception handlers are specified with the \keyword{try} ... \keyword{except}
statement.  The \keyword{try} ... \keyword{finally} statement
specifies cleanup code which does not handle the exception, but is
executed whether an exception occurred or not in the preceding code.

Python uses the ``termination'' \index{termination model}model of
error handling: an exception handler can find out what happened and
continue execution at an outer level, but it cannot repair the cause
of the error and retry the failing operation (except by re-entering
the offending piece of code from the top).

When an exception is not handled at all, the interpreter terminates
execution of the program, or returns to its interactive main loop.  In
either case, it prints a stack backtrace, except when the exception is 
\exception{SystemExit}\withsubitem{(built-in
exception)}{\ttindex{SystemExit}}.

Exceptions are identified by string objects or class instances.
Selection of a matching except clause is based on object identity
(i.e., two different string objects with the same value represent
different exceptions!)  For string exceptions, the \keyword{except}
clause must reference the same string object.  For class exceptions,
the \keyword{except} clause must reference the same class or a base
class of it.

When an exception is raised, an object (maybe \code{None}) is passed
as the exception's ``parameter'' or ``value''; this object does not
affect the selection of an exception handler, but is passed to the
selected exception handler as additional information.  For class
exceptions, this object must be an instance of the exception class
being raised.

See also the description of the \keyword{try} statement in section
\ref{try} and \keyword{raise} statement in section \ref{raise}.
