% Quick Reference Card for GNUPLOT 1992
% Format stolen shamelessly from the GNU Emacs reference card
%**start of header
\special{landscape}
\newcount\columnsperpage

% This file can be printed with 1, 2, or 3 columns per page (see below).
% Specify how many you want here.  Nothing else needs to be changed.
% For GNUPLOT refcard, entries are too wide for 3 columns.  Print
% 2 columns landscape

\columnsperpage=2

% Copyright (c) 1987 Free Software Foundation, Inc.

% This file is part of GNU Emacs, but was adapted for the GNUPLOT
% reference card because it was so nicely set up.

% This file is intended to be processed by plain TeX (TeX82).
%
% The final reference card has six columns, three on each side.
% This file can be used to produce it in any of three ways:
% 1 column per page
%    produces six separate pages, each of which needs to be reduced to 80%.
%    This gives the best resolution.
% 2 columns per page
%    produces three already-reduced pages.
%    You will still need to cut and paste.
% 3 columns per page
%    produces two pages which must be printed sideways to make a
%    ready-to-use 8.5 x 11 inch reference card.
%    For this you need a dvi device driver that can print sideways.
% Which mode to use is controlled by setting \columnsperpage above.
%
% Author of GNU Emacs Refcard:
%  Stephen Gildea
%  UUCP: mit-erl!gildea
%  Internet: gildea@erl.mit.edu
%
% Thanks to Paul Rubin, Bob Chassell, Len Tower, and Richard Mlynarik
% for their many good ideas.
%
% Person who ripped off the formatter:
%  Alex Woo
%  NASA Ames Research Center
%  Internet: woo@ames.arc.nasa.gov
%
% Modified on 9 Dec 1992 by:
%  Daniel S. Lewart
%  University of Illinois
%  Internet: d-lewart@uiuc.edu

% make \bye not \outer so that the \def\bye in the \else clause below
% can be scanned without complaint.
\def\bye{\par\vfill\supereject\end}

\def\copyrightnotice{
\vskip 1ex plus 2 fill\begingroup\small
Layout adapted from the \TeX\ source for Stephen Gildea's GNU Emacs
Reference Card (version 1.8).

PostScript is a registered trademark of Adobe Systems Incorporated.
GNU, Versatec, Imagen, Printronix, Canon, GraphOn, Visual, MicroVAX,
UIS, Tektronix, Unix, VAX, VMS, NeWS, SunView, and em4010 are all
trademarks of various companies.  We endorse none of them.

\endgroup}

\newdimen\intercolumnskip
\newbox\columna
\newbox\columnb

\def\ncolumns{\the\columnsperpage}

\message{[\ncolumns\space
  column\if 1\ncolumns\else s\fi\space per page]}

\def\scaledmag#1{ scaled \magstep #1}

% This multi-way format was designed by Stephen Gildea
% October 1986.
% modified for GNUPLOT refcard by Alex Woo

\if 1\ncolumns
%  \hsize 4in
  \hsize 6in
  \vsize 10in
  \voffset -.7in
  \font\titlefont=\fontname\tenbf \scaledmag3
  \font\headingfont=\fontname\tenbf \scaledmag2
  \font\smallfont=\fontname\sevenrm
  \font\smallsy=\fontname\sevensy

  \footline{\hss\folio}
  \def\makefootline{\baselineskip10pt\hsize6.5in\line{\the\footline}}
\else
%  \hsize 3.2in
  \hsize 5in
  \vsize 7.95in
  \hoffset -.75in
  \voffset -.745in
  \font\titlefont=cmbx10 \scaledmag2
  \font\headingfont=cmbx10 \scaledmag1
  \font\smallfont=cmr6
  \font\smallsy=cmsy6
  \font\eightrm=cmr8
  \font\eightbf=cmbx8
  \font\eightit=cmti8
  \font\eighttt=cmtt8
  \font\eightsy=cmsy8
  \textfont0=\eightrm
  \textfont2=\eightsy
  \def\rm{\eightrm}
  \def\bf{\eightbf}
  \def\it{\eightit}
  \def\tt{\eighttt}
  \normalbaselineskip=.8\normalbaselineskip
  \normallineskip=.8\normallineskip
  \normallineskiplimit=.8\normallineskiplimit
  \normalbaselines\rm           %make definitions take effect

  \if 2\ncolumns
    \let\maxcolumn=b
    \footline{\hss\rm\folio\hss}
    \def\makefootline{\vskip 2in \hsize=6.86in\line{\the\footline}}
  \else \if 3\ncolumns
    \let\maxcolumn=c
    \nopagenumbers
  \else
    \errhelp{You must set \columnsperpage equal to 1, 2, or 3.}
    \errmessage{Illegal number of columns per page}
  \fi\fi

  \intercolumnskip=.46in
  \def\abc{a}
  \output={%
      % This next line is useful when designing the layout.
      %\immediate\write16{Column \folio\abc\space starts with \firstmark}
      \if \maxcolumn\abc \multicolumnformat \global\def\abc{a}
      \else\if a\abc
        \global\setbox\columna\columnbox \global\def\abc{b}
        %% in case we never use \columnb (two-column mode)
        \global\setbox\columnb\hbox to -\intercolumnskip{}
      \else
        \global\setbox\columnb\columnbox \global\def\abc{c}\fi\fi}
  \def\multicolumnformat{\shipout\vbox{\makeheadline
      \hbox{\box\columna\hskip\intercolumnskip
        \box\columnb\hskip\intercolumnskip\columnbox}
      \makefootline}\advancepageno}
  \def\columnbox{\leftline{\pagebody}}

  \def\bye{\par\vfill\supereject
    \if a\abc \else\null\vfill\eject\fi
    \if a\abc \else\null\vfill\eject\fi
    \end}
\fi

% we won't be using math mode much, so redefine some of the characters
% we might want to talk about
\catcode`\^=12
\catcode`\_=12

\chardef\\=`\\
\chardef\{=`\{
\chardef\}=`\}

\hyphenation{mini-buf-fer}

\parindent 0pt
\parskip 1ex plus .5ex minus .5ex

\def\small{\smallfont\textfont2=\smallsy\baselineskip=.8\baselineskip}

\outer\def\newcolumn{\vfill\eject}

\outer\def\title#1{{\titlefont\centerline{#1}}\vskip 1ex plus .5ex}

\outer\def\section#1{\par\filbreak
  \vskip 3ex plus 2ex minus 2ex {\headingfont #1}\mark{#1}%
  \vskip 2ex plus 1ex minus 1.5ex}

\newdimen\keyindent

\def\beginindentedkeys{\keyindent=1em}
\def\endindentedkeys{\keyindent=0em}
\endindentedkeys

\def\paralign{\vskip\parskip\halign}

\def\<#1>{$\langle${\rm #1}$\rangle$}

\def\kbd#1{{\tt#1}\null}        %\null so not an abbrev even if period follows

\def\beginexample{\par\leavevmode\begingroup
  \obeylines\obeyspaces\parskip0pt\tt}
{\obeyspaces\global\let =\ }
\def\endexample{\endgroup}


\def\key#1#2{\leavevmode\hbox to \hsize{\vtop
%  {\hsize=.75\hsize\rightskip=1em
  {\hsize=.5\hsize\rightskip=1em
  \hskip\keyindent\relax#1}\kbd{#2}\hfil}}

\newbox\metaxbox
\setbox\metaxbox\hbox{\kbd{M-x }}
\newdimen\metaxwidth
\metaxwidth=\wd\metaxbox

\def\metax#1#2{\leavevmode\hbox to \hsize{\hbox to .75\hsize
  {\hskip\keyindent\relax#1\hfil}%
  \hskip -\metaxwidth minus 1fil
  \kbd{#2}\hfil}}

\def\threecol#1#2#3{\hskip\keyindent\relax#1\hfil&\kbd{#2}\quad
  &\kbd{#3}\quad\cr}

%**end of header


\title{GNUPLOT Quick Reference}

\centerline{(Copyright(c) Alex Woo 1992 June 1)}

\section{Starting GNUPLOT}

\key{to enter GNUPLOT}{gnuplot}
\key{to enter batch GNUPLOT}{gnuplot macro_file}
\key{to pipe commands to GNUPLOT}{application | gnuplot}

see below for  environment variables you might want to change
before entering GNUPLOT.

\section{Exiting GNUPLOT}

\key{exit GNUPLOT}{quit}

All GNUPLOT commands can be abbreviated to the first few
unique letters, usually three characters.  This reference uses
the complete name for clarity.

\section{Getting Help}

\key{introductory help} {help plot}
\key{help on a topic}{help <topic>}
\key{list of all help available}{help or ?}
\key{show current environment}{show all}
\section{Command-line Editing}

The UNIX, MS-DOS and VMS versions of GNUPLOT support command-line
editing and a command history.  EMACS style editing is supported.

\beginindentedkeys

Line Editing:

\key{move back a single character}{^ B}
\key{move forward a single character}{^ F}
\key{moves to the beginning of the line}{^ A}
\key{moves to the end of the line}{^ E}
\key{delete the previous character} {^ H and DEL }
\key{deletes the current character}  {^ D}
\key{deletes to the end of line}{^ K}
\key{redraws line in case it gets trashed}{ ^ L,^ R}
\key{deletes the entire line}{ ^ U}
\key{deletes the last word}{ ^ W}
\endindentedkeys
\beginindentedkeys

History:

\key{moves back through history}{ ^ P }
\key{moves forward through history}{ ^ N }
\endindentedkeys

The following arrow keys may be used on the MS-DOS version if READLINE is
used.

\beginindentedkeys
IBM PC Arrow Keys:

\key{Left  Arrow}{same as ^ B}
\key{Right Arrow}{same as ^ F}
\key{Ctrl Left  Arrow}{same as ^ A}
\key{Ctrl Right Arrow}{same as ^ E}
\key{Up    Arrow}{same as ^ P}
\key{Down  Arrow}{same as ^ N}
\endindentedkeys



\section{Graphics Devices}

All screen graphics devices are specified by names and options.
This information can be read from a startup file (.gnuplot in UNIX).
If  you change the graphics device, you must replot with the
\kbd{replot} command.

\key{get a list of valid devices }{set terminal [options]}

\beginindentedkeys

Graphics Terminals:

\key{AED 512 Terminal}                          {set term aed512}
\key{AED 767 Terminal}                          {set term aed767}
\key{Amiga}                                     {set term amiga}
\key{Adobe Illustrator 3.0 Format}              {set term aifm}
\key{Apollo graphics primitive, rescalable}     {set term apollo}
\key{Atari ST}                                  {set term atari}
\key{BBN Bitgraph Terminal}                     {set term bitgraph}
\key{SCO CGI Driver}                            {set term cgi}
\key{Apollo graphics primitive, fixed window}   {set term gpr}
\key{SGI GL window}                             {set term iris4d [8 24]}
\key{MS-DOS Kermit Tek4010 term - color}        {set term kc_tek40xx}
\key{MS-DOS Kermit Tek4010 term - mono}         {set term km_tek40xx}
\key{NeXTstep window system}                    {set term next}
\key{REGIS graphics language}                   {set term regis}
\key{Selanar Tek Terminal}                      {set term selanar}
\key{SunView window system}                     {set term sun}
\key{Tektronix 4106, 4107, 4109 \& 420X }       {set term tek4OD10x}
\key{Tektronix 4010; most TEK emulators}        {set term tek40xx}
\key{VAX UIS window system}                     {set term VMS}
\key{VT-like tek40xx terminal emulator}         {set term vttek}
\key{UNIX plotting (not always supplied)}       {set term unixplot}
\key{AT\&T 3b1 or 7300 UNIXPC}                  {set term unixpc}
\key{X11 default display device}                {set term x11}
\key{X11 multicolor point default device}       {set term X11}

\endindentedkeys

\beginindentedkeys

Turbo C PC Graphics Modes:

\key{Hercules}{set term hercules}
\key{Color Graphics Adaptor}{set term cga}
\key{Monochrome CGA}{set term mcga}
\key{Extended Graphics Adaptor}{set term ega}
\key{VGA} {set term vga}
\key{Monochrome VGA} {set term vgamono}
\key{Super VGA - requires SVGA driver}{set term svga}
\key{AT\&T 6300 Micro}{set term att}

\endindentedkeys

MS Windows 3.x and OS/2 Presentation Manager are also supported.

\beginindentedkeys

Hardcopy Devices:

\key{Unknown - not a plotting device} {set term unknown}
\key{Dump ASCII table of X Y [Z] values}{set term table}
\key{printer or glass dumb terminal}                    {set term dumb}
\key{Roland DXY800A plotter}                            {set term dxy800a}

Dot Matrix Printers

\key{Epson-style 60-dot per inch printers}              {set term epson_60dpi}
\key{Epson LX-800, Star NL-10 }{set term epson_lx800}
\key{NX-1000, PROPRINTER }{set term epson_lx800}
\key{NEC printer CP6, Epson LQ-800 }
                        {set term nec_cp6 [monochrome color draft]}
\key{Star Color Printer}                                {set term starc}
\key{Tandy DMP-130  60-dot per inch }   {set term tandy_60dpi}
\key{Vectrix 384 \& Tandy color printer}                {set term vx384}

Laser Printers

\key{Talaris EXCL language}{set term excl}
\key{Imagen laser printer}                              {set term imagen}
\key{LN03-Plus in EGM mode}                     {set term ln03}
\key{PostScript graphics language }
                {set term post [mode color `font' size]}
\key{CorelDraw EPS}
                {set term corel [mode color `font' size]}
\key{Prescribe - for the Kyocera Laser Printer} {set term prescribe}
\key{Kyocera Laser Printer with Courier font}           {set term kyo}
\key{QMS/QUIC Laser  (also Talaris 1200 )}{set term qms}

Metafiles

\key{AutoCAD DXF (120x80 default)}                      {set term dxf}
\key{FIG graphics language: SunView or X }{set term fig}
\key{FIG graphics language:  Large Graph}{set term bfig}
\key{SCO hardcopy CGI}{set term hcgi}
\key{Frame Maker MIF 3.0}
                {set term mif [pentype curvetype help]}
\key{Portable bitmap}                                   {set term pbm [fontsize color]}
\key{Uniplex Redwood Graphics Interface Protocol}{set term rgip}
\key{TGIF language}                             {set term tgif}

HP Devices

\key{HP2623A and maybe others}                  {set term hp2623A}
\key{HP2648 and HP2647}                         {set term hp2648}
\key{HP7580, \& probably other HPs (4 pens)}            {set term hp7580B}
\key{HP7475 \&  lots of others (6 pens)}        {set term hpgl}
\key{HP Laserjet series II \& clones}   {set term hpljii [75 100 150 300]}
\key{HP DeskJet 500} {set term hpdj [75 100 150 300]}
\key{HP PaintJet \& HP3630 }
                                {set term hppj [FNT5X9 FNT9X17 FNT13x25]}
\key{HP laserjet III ( HPGL plot vectors)}
                {set term pcl5 [mode font fontsize ]}

TeX picture environments

\key{LaTeX picture environment}                 {set term latex}
\key{EEPIC -- extended LaTeX picture }  {set term eepic}
\key{LaTeX picture  with emTeX specials}        {set term emtex}
\key{PSTricks macros for TeX or LaTeX}  {set term pstricks}
\key{TPIC specials for TeX or LaTeX}    {set term tpic}
\key{MetaFont font generation input}                    {set term mf}


\endindentedkeys

\section{Files}

\key{{\bf plot} a data file}{plot `fspec'}
\key{{\bf load} in a macro file}{load `fspec'}
\key{{\bf save} command buffer to a macro file}{save `fspec'}
\key{{\bf save settings} for later reuse}{save set `fpec'}

\section{PLOT \& SPLOT commands}


{\bf plot} and {\bf splot} are the primary commands
{\bf plot} is used to plot 2-d
functions and data, while {\bf splot} plots 3-d surfaces and data.

Syntax:

        plot $\{$ranges$\}$ $<$function$> \{$title$\} \{$style$\}$
                    $\{, <$function$> \{$title$\} \{$style$\}...\}$

        splot $\{$ranges$\} <$function$> \{$title$\} \{$style$\}$
                    $\{, <$function$> \{$title$\} \{$style$\}...\}$

where $<$function$>$ is either a mathematical expression, the name of a
data file enclosed in quotes, or a pair ({\bf plot}) or triple ({\bf splot})
of mathematical expressions in the case of parametric functions.
User-defined functions and variables may also be defined here.
Examples will be given below.

\section{Plotting Data}
Discrete data contained in a file can displayed by specifying the
name of the data file (enclosed in quotes) on the {\bf plot} or {\bf splot}
command line. Data files should contain one data point per line.
Lines beginning with \# (or ! on VMS) will be treated as comments
and ignored. For {\bf plot}s, each data point represents an (x,y)
pair. For {\bf splot}s, each point is an (x,y,z) triple. For {\bf plot}s with
error bars (see {\bf plot errorbars}), each data point is either
(x,y,ydelta), (x,y,ylow,yhigh), (x,y,xlow,xhigh), (x,y,xdelta,ydelta), or
(x,y,xlow,xhigh,ylow,yhigh). In all cases, the numbers on each
line of a data file must be separated by blank space. This blank
space divides each line into columns.

For {\bf plot}s the x value may be omitted, and for {\bf splot}s the x
and y values may be omitted. In either case the omitted values are
assigned the current coordinate number. Coordinate numbers start at 0
and are incremented for each data point read.

\section{Surface Plotting}
Implicitly, there are two types of 3-d datafiles. If all the isolines
are of the same length, the data is assumed to be a grid data, i.e.,
the data has a grid topology. Cross isolines in the other parametric
direction (the ith cross isoline passes thru the ith point of all the
provided isolines) will also be drawn for grid data. (Note contouring
is available for grid data only.) If all the isolines are not of the
same length, no cross isolines will be drawn and contouring that data
is impossible.

For splot if 3-d datafile and using format (see {\bf splot datafile using})
specify only z (height field), a non parametric mode must be specified.
If, on the other hand, x, y, and z are all specified, a parametric
mode should be selected (see {\bf set parametric}) since data is defining a
parametric surface.

\key{example of plotting a 3-d data}{set parametric;splot 'glass.dat'}
\key{example of plotting explicit}{set noparametric;splot 'datafile.dat'}

\section{Using Pipes}

On some computer systems with a popen function (UNIX), the datafile
can be piped through a shell command by starting the file name
with a '$<$'.  For example:

        pop(x) = 103*exp(x/10)
        plot "$<$ awk '$\{$ print \$1-1965 \$2 $\}$' population.dat", pop(x)

would plot the same information as the first population example
but with years since 1965 as the x axis.

Similarly, output can be piped to another application, e.g.

        set out "$|$lpr -Pmy\_laser\_printer"

\section{Plot Data Using}
The format of data within a file can be selected with the {\bf using}
option. An explicit scanf string can be used, or simpler column
choices can be made.

\key{plot "datafile"}{ $\{$ using $\{ <$ycol$> |$}
\key{}{$<$xcol$>:<$ycol$> |$}
\key{}{$<$xcol$>:<$ycol$>:<$ydelta$> |$}
\key{}{$<$xcol$>:<$ycol$>:<$width$> |$}
\key{}{$<$xcol$>:<$ycol$>:<$xdelta$> |$}
\key{}{$<$xcol$>:<$ycol$>:<$ylo$>:<$yhi$> |$}
\key{}{$<$xcol$>:<$ycol$>:<$xlo$>:<$xhi$> |$}
\key{}{$<$xcol$>:<$ycol$>:<$xdelta$>:<$ydelta$> |$}
\key{}{$<$xcol$>:<$ycol$>:<$ydelta$>:<$width$> |$}
\key{}{$<$xcol$>:<$ycol$>:<$ylo$>:<$yhi$>:<$width$> |$}
\key{}{$<$xc$>:<$yc$>:<$xlo$>:<$xhi$>:<$ylo$>:<$yhi$> \}$}
\key{}{$\{$"<scanf string>"$\} \} ...$}

\key{splot "datafile"}
{$\{$ using $\{ <$xcol$>:<$ycol$>:<$zcol$> \}$}
\key{}{$\{"<$scanf string$>"\} \} ...$}

$<$xcol$>$, $<$ycol$>$, and $<$zcol$>$ explicitly select the columns to plot from
a space or tab separated multicolumn data file. If only $<$ycol$>$ is
selected for {\bf plot}, $<$xcol$>$ defaults to 1. If only $<$zcol$>$ is selected
for {\bf splot}, then only that column is read from the file. An $<$xcol$>$ of
0 forces $<$ycol$>$ to be plotted versus its coordinate number. $<$xcol$>$,
$<$ycol$>$, and $<$zcol$>$ can be entered as constants or expressions.

If errorbars (see also {\bf plot errorbars}) are used for {\bf plot}s,
xdelta or ydelta (for example, a +/- error) should be provided as the third
column, or (x,y)low and (x,y)high as third and fourth columns.  These columns
must follow the x and y columns. If errorbars in both directions are wanted
then xdelta and ydelta should be in the third and fourth columns, respectively,
or xlow, xhigh, ylow, yhigh should be in the third, fourth, fifth, and sixth
columns, respectively.

Scanf strings override any $<$xcol$>$:$<$ycol$>$(:$<$zcol$>$) choices, except for
ordering of input, e.g.,

\key{plot "datafile"}{ using 2:1 "\%f\%*f\%f"}

causes the first column to be y and the third column to be x.

If the scanf string is omitted, the default is generated based on the
$<$xcol$>$:$<$ycol$>$(:$<$zcol$>$) choices. If the {\bf using} option is omitted, ''\%f\%f''
is used for {\bf plot} (''\%f\%f\%f\%f'' or ''\%f\%f\%f\%f\%f\%f'' for {\bf 
errorbar} {\bf plot}s) and ''\%f\%f\%f'' is used for {\bf splot}.

\key{plot "MyData"} {using "\%*f\%f\%*20[^$\backslash$n]\%f" w lines}

Data are read from the file ``MyData'' using the format
''\%*f\%f\%*20[^$\backslash$n]\%f''. The meaning of this format is: ''\%*f'' ignore the
first number, ''\%f'' then read in the second and assign to x,
''\%*20[^$\backslash$n]'' then ignore 20 non-newline characters, ''\%f'' then read in
the y value.

\section{Plot With Errorbars}
Error bars are supported for 2-d data file plots by reading one to
four additional columns specifying ydelta, ylow and yhigh, xdelta,
xlow and xhigh, xdelta and ydelta, or xlow, xhigh, ylow, and yhigh
respectively. No support exists for error bars for {\bf splot}s.

In the default situation, GNUPLOT expects to see three to six
numbers on each line of the data file, either (x, y, ydelta),
(x, y, ylow, yhigh), (x, y, xdelta), (x, y, xlow, xhigh), 
(x, y, xdelta, ydelta), or (x, y, xlow, xhigh, ylow, yhigh). The x 
coordinate must be specified. The order of the numbers must be exactly 
as given above. Data files in this format can easily be plotted with error 
bars:

        plot "data.dat" with errorbars (or yerrorbars)

        plot "data.dat" with xerrorbars

        plot "data.dat" with xyerrorbars

The error bar is a line plotted from (x, ylow) to (x,
yhigh) or (xlow, y) to (xhigh, y). If ydelta is specified instead 
of ylow and yhigh, ylow=y-ydelta and yhigh=y+ydelta are derived. The
values for xlow and xhigh are derived similarly from xdelta. If there
are only two numbers on the line, yhigh and ylow are both set to
y and xhigh and xlow are both set to x. To get lines plotted between 
the data points, {\bf plot} the data file twice, once with errorbars and 
once with lines.

If x or y autoscaling is on, the x or y range will be adjusted to fit the
error bars.

Boxes may be drawn with y error bars using the {\bf boxerrorbars} style. The
width of the box may be either set with the "set boxwidth" command, given in 
one of the data columns, or calculated automatically so each box touches the 
adjacent boxes.
Boxes may be drawn instead of the cross drawn for the {\bf xyerrorbars} style 
by using the {\bf boxxyerrorbars} style. 

\key{x,y,ylow \& yhigh from columns 1,2,3,4}{plot "data.dat" us 1:2:3:4 w errorbars}
\key{x from third, y from second, xdelta from 6}{plot "data.dat" using 3:2:6 w xerrorbars}
\key{x,y,xdelta \& ydelta from columns 1,2,3,4}{plot "data.dat" us 1:2:3:4 w
xyerrorbars}

\section{Plot Ranges}
The optional range specifies the region of the plot that will be
displayed.

Ranges may be provided on the {\bf plot} and {\bf splot} command line and
affect only that plot, or in the {\bf set xrange}, {\bf set yrange}, etc.,
commands, to change the default ranges for future plots.

\key{[$\{<$dummy-var$> =\} \{<$xmin$> : <$xmax$>\}$]} { $\{$ [$\{<$ymin$> : <$ymax$>\}$] $\}$}

where $<$dummy-var$>$ is the independent variable (the defaults are x and
y, but this may be changed with {\bf set dummy}) and the min and max
terms can be constant expressions.

Both the min and max terms are optional. The ':' is also optional
if neither a min nor a max term is specified. This allows '[ ]' to
be used as a null range specification.

Specifying a range in the {\bf plot} command line turns autoscaling for
that axis off for that plot. Using one of the {\bf set} range commands
turns autoscaling off for that axis for future plots, unless changed
later. (See {\bf set autoscale}).

\key{This uses the current ranges}{plot cos(x)}
\key{This sets the x range only}{plot [-10:30] sin(pi*x)/(pi*x)}
\key{This sets both the x and y ranges}{plot [-pi:pi] [-3:3]  tan(x), 1/x}
\key{sets only y range, \&} {plot [ ] [-2:sin(5)*-8] sin(x)**besj0(x)}
\key{turns off autoscaling on both axes}{}
\key{This sets xmax and ymin only}{plot [:200] [-pi:]  exp(sin(x))}
\key{This sets the x, y, and z ranges}{splot [0:3] [1:4] [-1:1] x*y}

\section{Plot With Style}
Plots may be displayed in one of twelve styles: {\bf lines}, {\bf points},
{\bf linespoints}, {\bf impulses}, {\bf dots}, {\bf steps},
{\bf errorbars} (or {\bf yerrorbars}), {\bf xerrorbars}, {\bf xyerrorbars},
{\bf boxes}, {\bf boxerrorbars}, or {\bf boxxyerrorbars}. The {\bf lines} style
connects adjacent points with lines. The {\bf points} style displays a
small symbol at each point. The {\bf linespoints} style does both
{\bf lines} and {\bf points}. The {\bf impulses} style displays a vertical line
from the x axis (or from the grid base for {\bf splot}) to each point. The
{\bf dots} style plots a tiny dot at each point; this is useful for
scatter plots with many points. The {\bf steps} style is used for drawing 
stairstep-like functions. The {\bf boxes} style may be used for barcharts.

The {\bf errorbars} style is only relevant to 2-d data file plotting. It
is treated like {\bf points} for {\bf splot}s and function {\bf plot}s. For 
data
{\bf plot}s, {\bf errorbars} is like {\bf points}, except that a vertical error
bar is also drawn: for each point (x,y), a line is drawn from
(x,ylow) to (x,yhigh). A tic mark is placed at the ends of the error
bar. The ylow and yhigh values are read from the data file's columns,
as specified with the {\bf using} option to plot. The {\bf xerrorbars}
style is similar except that it draws a horizontal error bar from xlow to
xhigh. The {\bf xyerrorbars} or {\bf boxxyerrorbars} style is used for data 
with errors in both x and y. A barchart style may be used in conjunction with y 
error bars through the use of {\bf boxerrorbars}. The See {\bf plot errorbars} 
for more information.

Default styles are chosen with the {\bf set function style} and
{\bf set data style} commands.

By default, each function and data file will use a different
line type and point type, up to the maximum number of available
types. All terminal drivers support at least six different point
types, and re-use them, in order, if more than six are required.
The LaTeX driver supplies an additional six point types (all variants
of a circle), and thus will only repeat after twelve curves are
plotted with points.

If desired, the style and (optionally) the line type and point type
used for a curve can be specified.

\key{with $<$style$>$}{$ \{<$linetype$> \{<$pointtype$>\}\}$}

where $<$style$>$ is either {\bf lines}, {\bf points}, {\bf linespoints}, {\bf impulses},
{\bf dots}, {\bf steps}, {\bf errorbars} (or {\bf yerrorbars}), 
{\bf xerrorbars}, {\bf xyerrorbars}, {\bf boxes}, {\bf boxerrorbars},
{\bf boxxyerrorbars}.

The $<$linetype$>$ \& $<$pointtype$>$ are positive
integer constants or expressions and specify the line type and point
type to be used for the plot. Line type 1 is the first line type used
by default, line type 2 is the second line type used by default, etc.

\key{plots sin(x) with impulses}{plot sin(x) with impulses}
\key{plots x*y with points, x**2 + y**2 default}{splot x*y w points, x**2 + y**2}
\key{plots tan(x) with default function style}
        {plot [ ] [-2:5] tan(x)}
\key{plots ``data.1'' with lines}{plot "data.1" with l}
\key{plots ``leastsq.dat'' with impulses} {plot 'leastsq.dat' w i}
\key{plots ``exper.dat'' with errorbars \& }
{plot 'exper.dat' w l, 'exper.dat' w err}
\key{ lines connecting points}{}

Here 'exper.dat' should have three or four data columns.

\key{plots x**2 + y**2 and x**2 - y**2 with the same line type}
{splot x**2 + y**2 w l 1, x**2 - y**2 w l 1}
\key{plots sin(x) and cos(x) with linespoints, using}
    {plot sin(x) w linesp 1 3, \\}
\key{ the same line type but different point types}{ cos(x) w linesp 1 4}
\key{plots file ``data'' with points style 3}
{plot "data" with points 1 3}

Note that the line style must be specified when specifying the point
style, even when it is irrelevant. Here the line style is 1 and the
point style is 3, and the line style is irrelevant.

See {\bf set style} to change the default styles.

\section{Plot Title}

A title of each plot appears in the key. By default the title is
the function or file name as it appears on the plot command line.
The title can be changed by using the {\bf title} option. This option
should precede any {\bf with} option.

\key{        title "$<$title$>$"}{}

where $<$title$>$ is the new title of the plot and must be enclosed in
quotes. The quotes will not be shown in the key.

\key{plots y=x with the title 'x'} {plot x}
\key{plots the ``glass.dat'' file}
{splot "glass.dat" tit 'revolution surface'}
\key{with the title 'revolution surface'}{}
\key{plots x squared with title ``x^2'' and ``data.1''}
    {plot x**2 t "x^2", \\}
\key{ with title 'measured data'}{ "data.1" t 'measured data'}

\section{Set-Show Commands}

\beginindentedkeys
\key{all commands below begin with set}{set}
\key{set mapping of polar angles}{angles [degrees|radians]}
\key{arrows from point to}{arrow [<tag>][from <sx>,<sy>,<sz>]}
\key{}{ [to <ex>,<ey>,<ez>][nohead]}
\key{force autoscaling of an axis}{autoscale [<axes>]}
\key{enter/exit parametric mode} {[no]parametric}
\key{display border}{[no]border}
\key{clip points/line near boundaries}{[no]clip <clip-type>}
\key{specify parameters for contour plots}{cntrparam
[spline][points][order][levels]}
\key{enable splot contour plots}{[no]contour [base|surface|both]}
\key{default plotting style for data}{data style <style-choice>}
\key{specify dummy variable}{dummy <dummy1>,<dummy2>...}
\key{tic-mark label format specification}{format
[<axes>]["format-string"]}
\key{function plotting style}{function style <style-choice>}
\key{draw a grid at major tick marks \& minor tics (optional)}{[no]grid [mxgrid 
OR mygrid]}
\key{enables hiddenline removal}{[no]hidden3d}
\key{specify number of isolines}{isosamples <expression>}
\key{enables key of curves in plot}{key <x>,<y>,<z>}
\key{logscaling of an axes (optionally giving base)}{logscale <axes> [<base>]}
\key{mapping 3D coordinates}{mapping
[cartesian|spherical|cylindrical]}
\key{offsets from center of graph}{offsets
<left>,<right>,<top>,<bottom>}
\key{mapping 2D coordinates}{[no]polar}
\key{set radial range}{rrange [<rmin>:<rmax>]}
\key{set sampling rate of functions}{samples <expression>}
\key{set scaling factors of plot}{size <xsize>,<ysize>}
\key{control display of isolines of surface}{[no]surface}
\key{control graphics device}{terminal <device>}
\key{change direction of tics}{tics <direction>}
\key{adjust relative height of vertical axis}{ticslevel <level>}
\key{adjust size of tick marks}{ticscale [<size>]}
\key{turn on time/date stamp}{[no]time}
\key{set centered plot title}{title "title-text" <xoff>,<yoff>}
\key{set parametric range}{trange [<tmin>:<tmax>]}
\key{set surface parametric ranges}{urange or vrange}
\key{sets the view point for {\bf splot}}{view
<rot_x>,<rot_z>,<scale>,<scale_z>}
\key{sets x-axis label}{xlabel "<label>" <xoff>,<yoff>}
\key{set horizontal range}{xrange [<xmin>:<xmax>]}
\key{change horizontal tics}{xtics <start>,<incr>,<end>,}
\key{}{"<label>" <pos> }
\key{adjust number of minor tick marks}{[no]mxtics OR [no]mytics [<freq>]}
\key{draw x-axis}{[no]xzeroaxis}
\key{sets y-axis label}{ylabel "<label>" <xoff>,<yoff>}
\key{set vertical range}{yrange [<ymin>:<ymax>]}
\key{change vertical tics}{ytics <start>,<incr>,<end>,}
\key{}{"<label>" <pos> }
\key{draw y-axis}{[no]yzeroaxis}
\key{set default threshold for values near 0}{zero <expression>}
\key{draw axes}{[no]zeroaxis}
\key{sets z-axis label}{zlabel "<label>" <xoff>,<yoff>}
\key{set vertical range}{zrange [<zmin>:<zmax>]}
\key{change vertical tics}{ztics <start>,<incr>,<end>,}
\key{}{"<label>" <pos> }
\key{draw z-axis}{[no]zzeroaxis}
\endindentedkeys

\section{Contour Plots}
Enable contour drawing for surfaces. This option is available for {\bf splot}
only.

Syntax:
        set contour $\{$ base $|$ surface $|$ both $\}$
        set nocontour

If no option is provided to {\bf set contour}, the default is {\bf base}.
The three options specify where to draw the contours: {\bf base} draws
the contours on the grid base where the x/ytics are placed, {\bf surface}
draws the contours on the surfaces themselves, and {\bf both} draws the
contours on both the base and the surface.

See also {\bf set cntrparam} for the parameters that affect the drawing of
contours.

\section{Contour Parameters}
Sets the different parameters for the contouring plot (see also {\bf contour}).

 \key{set cntrparam}{ $\{ \{$ linear $|$ cubicspline $|$ bspline $\} |$}
\key{}{points $<$n$>$ $|$ }
\key{}{order $<$n$>$  $|$ }
\key{}{levels \{ [ auto ] $<$n$>$ $|$ }
\key{}{discrete $<$z1$>$ $<$z2$>$ ... $|$ }
\key{}{incr $<$start$>$ $<$increment$>$ [ $<$n$>$ ] $\} \}$ }

\key{5 automatic levels}{set cntrparam levels auto 5}
\key{3 discrete levels at 10\%, 37\% and 90\%}
        {set cntrp levels discrete .1 1/exp(1) .9}
\key{5 incremental levels at 0, .1, .2, .3 and .4}
        {set cntrparam levels incremental  0 .1 5 }
\key{sets n = 10 retaining current setting of auto, incr., or discr.}
        {set cntrparam levels 10 }
\key{set start = 100 and increment = 50, retaining old n}
        {set cntrparam levels incremental 100 50}

This command controls the way contours are plotted. $<$n$>$ should be an
integral constant expression and $<$z1$>$, $<$z2$>$ any constant expressions.
The parameters are:

{\bf linear}, {\bf cubicspline}, {\bf bspline} - Controls type of approximation or
interpolation. If {\bf linear}, then the contours are drawn piecewise
linear, as extracted from the surface directly. If {\bf cubicspline}, then
piecewise linear contours are interpolated to form a somewhat smoother
contours, but which may undulate. The third option is the uniform
{\bf bspline}, which only approximates the piecewise linear data but is
guaranteed to be smoother.

{\bf points} - Eventually all drawings are done with piecewise linear
strokes.  This number controls the number of points used to
approximate a curve.  Relevant for {\bf cubicspline} and {\bf bspline} modes
only.

{\bf order}  - Order of the bspline approximation to be used. The bigger this
order is, the smoother the resulting contour.  (Of course, higher order
bspline curves will move further away from the original piecewise linear
data.)  This option is relevant for {\bf bspline} mode only. Allowed values are
integers in the range from 2 (linear) to 10.

{\bf levels} - Number of contour levels, 'n'.  Selection of the levels is
controlled by 'auto' (default), 'discrete', and 'incremental'. For 'auto',
if the surface is bounded by zmin and zmax then contours will be
generated from zmin+dz to zmax-dz in steps of size dz, where
dz = (zmax - zmin) / (levels + 1).  For 'discrete', contours will be
generated at z = z1, z2 ... as specified.  The number of discrete levels
is limited to MAX\_DISCRETE\_LEVELS, defined in plot.h to be 30.  If
'incremental', contours are generated at $<$n$>$ values of z beginning at
$<$start$>$ and increasing by $<$increment$>$.

\section{Specifying Labels}
Arbitrary labels can be placed on the plot using the {\bf set label}
command.  If the z coordinate is given on a {\bf plot} it is ignored; if
it is missing on a {\bf splot} it is assumed to be 0.

\key{set label $\{<$tag$>\} \{"<$label_text$>"\}$}
{$\{$at $<$x$>,<$y$>\{,<$z$>\}\}$}
\key{}{$\{<$justification$>\}$}
\key{set nolabel $\{<$tag$>\}$}{}
\key{show label}{}

The text defaults to '''', and the position to 0,0,0.  The $<$x$>$, $<$y$>$, and
$<$z$>$ values are in the graph's coordinate system.  The tag is an
integer that is used to identify the label. If no $<$tag$>$ is given, the
lowest unused tag value is assigned automatically. The tag can be used
to delete or change a specific label. To change any attribute of an
existing label, use the {\bf set label} command with the appropriate tag,
and specify the parts of the label to be changed.

By default, the text is placed flush left against the point x,y,z.
To adjust the way the label is positioned with respect to the point
x,y,z, add the parameter $<$justification$>$, which may be {\bf left}, {\bf right}
or {\bf center}, indicating that the point is to be at the left, right or
center of the text. Labels outside the plotted boundaries are
permitted but may interfere with axes labels or other text.

\key{label at (1,2) to ``y=x'' }{set label "y=x" at 1,2}
\key{label ``y=x^2'' w right of the text at (2,3,4), }
{set label 3 "y=x^2" at 2,3,4 right}
\key{\& tag the label number 3 }{}
\key{change preceding label to center justification}{set label 3 center}
\key{delete label number 2 }{set nolabel 2}
\key{delete all labels}{set nolabel}
\key{show all labels (in tag order)}{show label}

(The EEPIC, Imagen, LaTeX, and TPIC drivers allow $\backslash$$\backslash$
in a string to specify a newline.)

\section{Miscellaneous Commands}

For further information on these commands, print out a copy
of the GNUPLOT manual.

\key{change working directory}{cd}
\key{erase current screen or device}{clear}
\key{exit GNUPLOT}{exit or quit or EOF}
\key{display text and wait}{pause <time> ["<string>"]}
\key{print the value of $<$expression$>$}{print <expression>}
\key{print working directory}{pwd}
\key{repeat last {\bf plot} or {\bf splot}}{replot}
\key{spawn an interactive shell}{! (UNIX) or \$ (VMS)}

\section{Environment Variables}

A number of shell environment variables are understood by GNUPLOT.
None of these are required, but may be useful.

If GNUTERM is defined, it is used as the name of the terminal type to
be used. This overrides any terminal type sensed by GNUPLOT on start
up, but is itself overridden by the .gnuplot (or equivalent) start-up
file (see {\bf start-up}), and of course by later explicit changes.

On Unix, AmigaOS, and MS-DOS, GNUHELP may be defined to be the pathname
of the HELP file (gnuplot.gih).

On VMS, the symbol GNUPLOT\$HELP should be defined as the name of
the help library for GNUPLOT.

On Unix, HOME is used as the name of a directory to search for
a .gnuplot file if none is found in the current directory.
On AmigaOS and MS-DOS, GNUPLOT is used. On VMS, SYS\$LOGIN: is used.
See help start-up.

On Unix, PAGER is used as an output filter for help messages.

On Unix and AmigaOS, SHELL is used for the {\bf shell} command. On MS-DOS,
COMSPEC is used for the {\bf shell} command.

On AmigaOS, GNUFONT is used for the screen font.  For example:
``setenv GNUFONT sapphire/14''.

On MS-DOS, if the BGI interface is used, the variable {\bf BGI} is used to point
to the full path to the BGI drivers directory. Furthermore SVGA is used to
name the Super VGA BGI driver in 800x600 res., and its mode of operation
as 'Name.Mode'.
For example, if the Super VGA driver is C:$\backslash$TC$\backslash$BGI$\backslash$SVGADRV.BGI and mode 3 is
used for 800x600 res., then: 'set BGI=C:$\backslash$TC$\backslash$BGI' and 'set SVGA=SVGADRV.3'.


\section{Expressions}
In general, any mathematical expression accepted by C, FORTRAN,
Pascal, or BASIC is valid. The precedence of these operators is
determined by the specifications of the C programming language.
White space (spaces and tabs) is ignored inside expressions.

Complex constants may be expressed as $\{<$real$>,<$imag$>\}$, where
$<$real$>$ and $<$imag$>$ must be numerical constants.  For example,
$\{3,2\}$ represents 3 + 2{\bf i} and $\{0,1\}$ represents {\bf i}
itself.  The curly braces are explicitly required here.
\section{Functions}
The functions in GNUPLOT are the same as the corresponding functions
in the Unix math library, except that all functions accept integer,
real, and complex arguments, unless otherwise noted. The {\bf sgn}
function is also supported, as in BASIC.

%\begin{center}
%\begin{tabular}{|ccl|} \hline
\settabs 5\columns
\+Function & Arguments & Returns \cr
\hrule%\hline
\+ abs(x) & any  &  absolute value of {\tt x}, $|x|$; same type \cr
\+ abs(x) & complex &  length of {\tt x}, $\sqrt{{\hbox{real}(x)^{2} +
\hbox{imag}(x)^{2}}}$ \cr
\+ acos(x) & any  & $\cos^{-1} x$ (inverse cosine) in radians \cr
\+ arg(x) & complex & the phase of $x$ in radians\cr
\+ asin(x) & any  & $\sin^{-1} x$ (inverse sin) in radians \cr
\+ atan(x) & any  & $\tan^{-1} x$ (inverse tangent) in radians \cr
\+ besj0(x) & radians &  $j_{0}$ Bessel function of $x$ \cr
\+ besj1(x) & radians & $j_{1}$ Bessel function of $x$ \cr
\+ besy0(x) & radians & $y_{0}$ Bessel function of $x$ \cr
\+ besy1(x) & radians & $y_{1}$ Bessel function of $x$ \cr
\+ ceil(x) & any & $\lceil x \rceil$, smallest integer not less than $x$
(real part) \cr
\+ cos(x) & radians & $\cos x$, cosine of $x$ \cr
\+ cosh(x) & radians & $\cosh x$, hyperbolic cosine of $x$ \cr
\+ erf(x) & any & $\hbox{Erf}(\hbox{real}(x))$,  error function of real($x$) \cr
\+ erfc(x) & any & $\hbox{Erfc}(\hbox{real}(x))$,  1.0 - error function of real($x$) \cr
\+ exp(x) & any & $e^{x}$,  exponential function of $x$ \cr
\+ floor(x) & any & $\lfloor x \rfloor$,  largest integer not greater
than $x$ (real part) \cr
\+ gamma(x) & any & $\hbox{Gamma}(\hbox{real}(x))$,  gamma function of real($x$) \cr
\+ ibeta(p,q,x) & any & $\hbox{Ibeta}(\hbox{real}(p,q,x))$,  ibeta function of real($p$,$q$,$x$) \cr
\+ igamma(a,x) & any & $\hbox{Igamma}(\hbox{real}(a,x))$,  igamma function of real($a$,$x$) \cr
\+ imag(x) & complex &  imaginary part of $x$ as a real number \cr
\+ int(x) & real &  integer part of $x$, truncated toward zero \cr
\+ lgamma(x) & any & $\hbox{Lgamma}(\hbox{real}(x))$,  lgamma function of real($x$) \cr
\+ log(x) & any & $\log_{e} x$,  natural logarithm (base $e$) of $x$ \cr
\+ log10(x) & any & $\log_{10} x$,  logarithm (base $10$) of $x$ \cr
\+ rand(x) & any & $\hbox{Rand}(\hbox{real}(x))$,  pseudo random number generator \cr
\+ real(x) & any &  real part of $x$ \cr
\+ sgn(x) & any & 1 if $x>0$, -1 if $x<0$, 0 if $x=0$. imag($x$) ignored \cr
\+ sin(x) & radians & $\sin x$, sine of $x$ \cr
\+ sinh(x) & radians & $\sinh x$, hyperbolic sine $x$ \cr
\+ sqrt(x) & any & $\sqrt{x}$,  square root of $x$ \cr
\+ tan(x) & radians & $\tan x$,  tangent of $x$ \cr
\+ tanh(x) & radians & $\tanh x$, hyperbolic tangent of $x$\cr
\hrule % \hline
%\end{tabular}
%\end{center}
\section{Operators}
The operators in GNUPLOT are the same as the corresponding operators
in the C programming language, except that all operators accept
integer, real, and complex arguments, unless otherwise noted.
The ** operator (exponentiation) is supported, as in FORTRAN.

Parentheses may be used to change order of evaluation.



%%%%%%%%%%%%%%%%%%%%%%%%%%%%%%%%%%%%%%%%%%%%%%%%%%%%%%%%%%%%
\bye
\subsubsection{Binary}
The following is a list of all the binary operators and their
usages:


\begin{center}
\begin{tabular}{|ccl|} \hline
\multicolumn{3}{|c|}{Binary Operators} \\
Symbol & Example & Explanation \\ \hline
\verb~**~ & \verb~a**b~ & exponentiation\\
\verb~*~ & \verb~a*b~ & multiplication\\
\verb~/~ & \verb~a/b~ & division\\
\verb~%~ & \verb~a%b~ & * modulo\\
\verb~+~ & \verb~a+b~ & addition\\
\verb~-~ & \verb~a-b~ & subtraction\\
\verb~==~ & \verb~a==b~ & equality\\
\verb~!=~ & \verb~a!=b~ & inequality\\
\verb~&~ & \verb~a&b~ & * bitwise AND\\
\verb~^~ & \verb~a^b~ & * bitwise exclusive OR\\
\verb~|~ & \verb~a|b~ & * bitwise inclusive OR\\
\verb~&&~ & \verb~a&&b~ & * logical AND\\
\verb~||~ & \verb~a||b~ & * logical OR\\
\verb~?:~ & \verb~a?b:c~ & * ternary operation\\
\hline
\end{tabular}
\end{center}
(*) Starred explanations indicate that the operator requires
integer arguments.

Logical AND (\&\&) and OR ($|$$|$) short-circuit the way they do in C.
That is, the second \&\& operand is not evaluated if the first is
false; the second $|$$|$ operand is not evaluated if the first is true.

The ternary operator evaluates its first argument (a). If it is
true (non-zero) the second argument (b) is evaluated and returned,
otherwise the third argument (c) is evaluated and returned.
\subsubsection{Unary}
The following is a list of all the unary operators and their
usages:


\begin{center}
\begin{tabular}{|ccl|} \hline
\multicolumn{3}{|c|}{Unary Operators}\\
Symbol & Example & Explanation \\ \hline
\verb@-@ & \verb@-a@ & unary minus \\
\verb@~@ & \verb@~a@ & * one's complement \\
\verb@!@ & \verb@!a@ & * logical negation \\
\verb@!@ & \verb@a!@ & * factorial \\
\hline
\end{tabular}
\end{center}
(*) Starred explanations indicate that the operator requires an
integer argument.

The factorial operator returns a real number to allow a greater range.
%%%%%%%%%%%%%%%%%%%%%%%%%%%%%%%%%%%%%%%%%%%%%%%%%5
